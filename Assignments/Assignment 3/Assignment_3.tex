\documentclass[fleqn, a4paper, 12pt, oneside]{amsart}
\usepackage{exsheets, tasks}
\usepackage{amsmath, amssymb, amsthm} %standard AMS packages
\usepackage{marginnote} %marginnotes
\usepackage{gensymb} %miscellaneous symbols
\usepackage{commath} %differential symbols
\usepackage{xcolor} %colours
\usepackage{cancel} %cancelling terms
\usepackage{siunitx} %formatting units
\usepackage{tikz, pgfplots} %diagrams
\usetikzlibrary{calc, hobby, patterns, intersections}
\usepackage{graphicx} %inserting graphics
\usepackage{hyperref} %hyperlinks
\usepackage{datetime} %date and time
\usepackage{ulem} %underline for \emph{}
\usepackage{xfrac} %inline fractions
\usepackage{enumerate, enumitem} %numbered lists
\usepackage{float} %inserting floats

\newcommand\numberthis{\addtocounter{equation}{1}\tag{\theequation}} %adds numbers to specific equations in non-numbered list of equations

\newcommand{\AxisRotator}[1][rotate=0]{
	\tikz [x=0.25cm,y=0.60cm,line width=.2ex,-stealth,#1] \draw (0,0) arc (-150:150:1 and 1);%
} %rotation symbols on axes

\theoremstyle{definition}
\newtheorem{example}{Example}
\newtheorem{definition}{Definition}

\theoremstyle{theorem}
\newtheorem{theorem}{Theorem}

\newcommand{\curl}{\mathrm{curl\,}}

\makeatletter
\@addtoreset{section}{part} %resets section numbers in new part
\makeatother

\renewcommand{\thesubsection}{(\arabic{subsection})}
\renewcommand{\thesection}{(\arabic{section})}

%section headings on left
\makeatletter
\def\specialsection{\@startsection{section}{1}%
	\z@{\linespacing\@plus\linespacing}{.5\linespacing}%
	%  {\normalfont\centering}}% DELETED
	{\normalfont}}% NEW
\def\section{\@startsection{section}{1}%
	\z@{.7\linespacing\@plus\linespacing}{.5\linespacing}%
	%  {\normalfont\scshape\centering}}% DELETED
	{\normalfont\scshape}}% NEW
\makeatother

%forces newline after subsection
\makeatletter
\def\subsection{\@startsection{subsection}{3}%
	\z@{.5\linespacing\@plus.7\linespacing}{.1\linespacing}%
	{\normalfont\itshape}}
\makeatother

\settasks{counter-format = tsk[1].}

\SetupExSheets{solution/print = true}

%opening
\title
{
	Differential and Integral Calculus\\
	Assignment 3
}
\author
{
	Aakash Jog\\
	ID : 989323563
}
\date{\formatdate{16}{4}{2015}}

\begin{document}
	
\maketitle
%\setlength{\mathindent}{0pt}

\begin{question}
	Prove or disprove:
	\begin{enumerate}[leftmargin=*]
		\item There exist two sequence $\{a_n\}$, $\{b_n\}$ such that $b_n \to -\infty$ and $a_n + b_n \to \infty$.
		\item If $a_n$ and $b_n$ are divergent sequences, then $a_n b_n$ is divergent.
		\item If $\{a_n\}$ has a subsequence that tends to infinity and $\{b_n\}$ has a subsequence that tends to infinity, then
$a_n + b_n$ is divergent.
		\item If $a_n$ is a convergent sequence, then $\lim_{n \to \infty} (a_{n + 1} - a_n) = 0$.
	\end{enumerate}
\end{question}

\begin{solution}
	\begin{enumerate}[leftmargin=*]
		\item 
			Let
			\begin{align*}
				a_n &= 2 n\\
				b_n &= -n
			\end{align*}
			Therefore,
			\begin{align*}
				\lim\limits_{n \to \infty} b_n &= -\infty\\
				\lim\limits_{n \to \infty} a_n + b_n &= \lim\limits_{n \to \infty} n\\
				&= \infty
			\end{align*}
			\qed
		\item
			Let
			\begin{align*}
				a_n &= (-1)^n\\
				b_n &= (-1)^n
			\end{align*}
			Therefore
			\begin{align*}
				a_n b_n &= (-1)^{2 n}\\
				&= 1
			\end{align*}
			Therefore $a_n b_n$ converges.\\
			Therefore the statement is false.
		\item
			Let $k \in \mathbb{N}$.\\
			Let
			\begin{align*}
				a_n &= 
					\begin{cases}
						n &;\quad n = 2 k\\
						-n &;\quad n \neq 2 k\\
					\end{cases}\\
				b_n &= 
					\begin{cases}
						-n &;\quad n = 2 k\\
						n &;\quad n \neq 2 k\\
					\end{cases}
			\end{align*}
			Therefore,
			\begin{align*}
				a_n + b_n &= 
					\begin{cases}
						n + (-n) &;\quad n = 2 k\\
						-n + n &;\quad n \neq 2 k\\
					\end{cases}\\
				&= 0
			\end{align*}
			Therefore $a_n + b_n$ converges.\\
			Therefore the statement is false.
		\item
			Let
			\begin{align*}
				\lim\limits_{n \to \infty} a_n &= l
			\end{align*}
			Therefore,
			\begin{align*}
				\lim\limits_{n \to \infty} a_{n + 1} - a_n &= \lim\limits_{n \to \infty} a_{n + 1} - \lim\limits_{n \to \infty} a_n\\
				&= l - l\\
				&= 0
			\end{align*}
			\qed
	\end{enumerate}
\end{solution}

\begin{question}
	Let $\{a_n\}$ be a sequence.
	Prove that if the subsequences $a_{2 k}$ and $a_{2 k + 1}$ converge to the same limit $L$ then $\lim\limits_{n \to \infty} a_n = L$.
\end{question}

\begin{solution}
	\begin{align*}
		\{a_n\} &= a_1, a_2, \dots, a_{2 k}, a_{2 k + 1}, \dots\\
		\therefore, \{a_n\} &= \{a_k, a_{2 k + 1}\}
	\end{align*}
	Therefore, as $n \to \infty$, if the even terms of $\{a_n\}$ and the odd terms of $\{a_n\}$ all tend to $L$, then $\{a_n\}$ itself tends to $L$.
	\qed
\end{solution}

\begin{question}
	Show that if $\{a_n\}$ is a sequence that is unbounded from above (i.e $\forall M > 0$ there exists $n \in \mathbb{N}$ such that $a_n > M$) then there exists a subsequence $\{a_{n_k}\}$ such that $\lim\limits_{k \to \infty} a_{n_k} = \infty$.
\end{question}

\begin{solution}
	If possible, $\nexists \{a_{n_k}\}$ such that $\lim\limits_{k \to \infty} a_{n_k} = \infty$.\\
	Therefore, there must be a maximum term in $\{a_n\}$, say $a_p$.\\
	Therefore, $\{a_n\}$ is bounded from above by any number greater than or equal to $a_p$.\\
	However, this contradicts the assumption that $\{a_n\}$ is unbounded from above.
	Therefore, there must exist a subsequence $\{a_{n_k}\}$ which tends to infinity.
	\qed
\end{solution}

\begin{question}
	Let $\{a_n\}$ be a sequence.
	Show that if $a_{n + 1} a_n \le 0$, $\forall n \in \mathbb{N}$ and the limit $\lim\limits_{n \to \infty} a_n$ exists, then $\lim\limits_{n \to \infty} a_n = 0$.
\end{question}

\begin{solution}
	\begin{align*}
		a_{n + 1} a_n &\le 0
	\end{align*}
	Therefore, either $a_{n + 1}$ and $a_n$ must have opposite parity, or at least one must be zero.\\
	~\\
	If $\forall n \in \mathbb{N}$, $a_{n + 1}$ and $a_n$ have opposite parity, then $\{a_n\}$ diverges.
	This contradicts the existence of $\lim\limits_{n \to \infty} a_n$.
	Therefore at least one of them must be zero.\\
	Therefore $\lim\limits_{n \to \infty} a_n = 0$.
\end{solution}

\begin{question}
	Give an example of a sequence $\{a_n\}_{n = 1}^{\infty}$ that satisfies $\lim\limits_{n \to \infty} (a_{n + 1} - a_n) = 0$, but the limit $\lim\limits_{n \to \infty} a_n$ does not exist (in the strict sense).
\end{question}

\begin{solution}
	Let
	\begin{align*}
		a_n &= \ln n\\
		\therefore \lim\limits_{n \to \infty} a_{n + 1} - a_n &= \lim\limits_{n \to \infty} \ln (n + 1) - \ln n\\
		&= \lim\limits_{n \to \infty} \ln \left( \dfrac{n + 1}{n} \right)\\
		&= \lim\limits_{n \to \infty} \ln \left( 1 + \dfrac{1}{n} \right)\\
		&= 0
	\end{align*}
	\begin{align*}
		\lim\limits_{n \to \infty} a_n &= \lim\limits_{n \to \infty} \ln n\\
		&= \infty
	\end{align*}
	Therefore, $\lim\limits_{n \to \infty} a_{n + 1} - a_n = 0$, but $\lim\limits_{n \to \infty} a_n$ does not exist.
\end{solution}

\begin{question}
	Find the following limits:
	\begin{enumerate}[leftmargin=*]
		\item $\lim\limits_{n \to \infty} \left( 1 + \dfrac{x}{n} \right)^n$ for $x \in \mathbb{R}$.
		\item $\lim\limits_{n \to \infty} n \tan^{-1} \left( \dfrac{1}{n} \right)$
	\end{enumerate}
\end{question}

\begin{solution}
	\begin{enumerate}[leftmargin=*]
		\item 
			\begin{align*}
				\lim\limits_{n \to \infty} \left( 1 + \dfrac{x}{n} \right)^n &= \lim\limits_{n \to \infty} \left( 1 + \dfrac{x}{n} \right)^{\frac{n}{x} \cdot x}\\
				&= \lim\limits_{n \to \infty} \left( \left( 1 + \dfrac{x}{n} \right)^{\frac{n}{x}} \right)^x\\
				&= e^x
			\end{align*}
		\item
			\begin{align*}
				\tan^{-1} x &= x - \dfrac{x^3}{3} + \dfrac{x^5}{5} - \dfrac{x^7}{7} + \dots\\
				\therefore \tan^{-1} \left( \dfrac{1}{n} \right) &= \dfrac{1}{n} - \dfrac{1}{3 n^3} + \dfrac{1}{5 n^5} - \dfrac{1}{7 n^7} + \dots\\
				\therefore n \tan^{-1} \left( \dfrac{1}{n} \right) &= n \left( \dfrac{1}{n} - \dfrac{1}{3 n^3} + \dfrac{1}{5 n^5} - \dfrac{1}{7 n^7} + \dots \right)\\
				&= 1 - \dfrac{1}{3 n^2} + \dfrac{1}{5 n^4} - \dfrac{1}{7 n^6} + \dots\\
				&= \sum\limits_{i = 0}^{\infty} \dfrac{(-1)^i}{(2 i + 1) n^{2 i}}
			\end{align*}
	\end{enumerate}
\end{solution}

\end{document}
