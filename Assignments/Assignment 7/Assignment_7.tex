\documentclass[fleqn, a4paper, 11pt, oneside]{amsart}
\usepackage{exsheets, tasks}
\usepackage{amsmath, amssymb, amsthm} %standard AMS packages
\usepackage{marginnote} %marginnotes
\usepackage{gensymb} %miscellaneous symbols
\usepackage{commath} %differential symbols
\usepackage{xcolor} %colours
\usepackage{cancel} %cancelling terms
\usepackage{siunitx} %formatting units
\usepackage{tikz, pgfplots} %diagrams
\usetikzlibrary{calc, hobby, patterns, intersections}
\usepackage{graphicx} %inserting graphics
\usepackage{hyperref} %hyperlinks
\usepackage{datetime} %date and time
\usepackage{ulem} %underline for \emph{}
\usepackage{xfrac} %inline fractions
\usepackage{enumerate, enumitem} %numbered lists
\usepackage{float} %inserting floats

\newcommand\numberthis{\addtocounter{equation}{1}\tag{\theequation}} %adds numbers to specific equations in non-numbered list of equations

\newcommand{\AxisRotator}[1][rotate=0]{
	\tikz [x=0.25cm,y=0.60cm,line width=.2ex,-stealth,#1] \draw (0,0) arc (-150:150:1 and 1);%
} %rotation symbols on axes

\theoremstyle{definition}
\newtheorem{example}{Example}
\newtheorem{definition}{Definition}

\theoremstyle{theorem}
\newtheorem{theorem}{Theorem}

\newcommand{\curl}{\mathrm{curl\,}}

\makeatletter
\@addtoreset{section}{part} %resets section numbers in new part
\makeatother

\renewcommand{\thesubsection}{(\arabic{subsection})}
\renewcommand{\thesection}{(\arabic{section})}

%section headings on left
\makeatletter
\def\specialsection{\@startsection{section}{1}%
	\z@{\linespacing\@plus\linespacing}{.5\linespacing}%
	%  {\normalfont\centering}}% DELETED
	{\normalfont}}% NEW
\def\section{\@startsection{section}{1}%
	\z@{.7\linespacing\@plus\linespacing}{.5\linespacing}%
	%  {\normalfont\scshape\centering}}% DELETED
	{\normalfont\scshape}}% NEW
\makeatother

%forces newline after subsection
\makeatletter
\def\subsection{\@startsection{subsection}{3}%
	\z@{.5\linespacing\@plus.7\linespacing}{.1\linespacing}%
	{\normalfont\itshape}}
\makeatother

\settasks{counter-format = tsk[1].}

\SetupExSheets{solution/print = true}

%opening
\title
{
	Differential and Integral Calculus\\
	Assignment 7
}
\author
{
	Aakash Jog\\
	ID : 989323563
}
\date{\formatdate{21}{5}{2015}}

\begin{document}
	
\maketitle
%\setlength{\mathindent}{0pt}

\begin{question}
	Check pointwise and uniform convergence of the following series of functions
	\begin{enumerate}
		\item $\sum\limits_{n = 0}^{\infty} \left( x^{n + 1} - x^n \right)$ in $[0,1]$.
		\item $\sum\limits_{n = 0}^{\infty} x^n$ in $[0,1]$.
		\item $\sum\limits_{n = 1}^{\infty} \frac{(-1)^n}{x^2 + n}$ in $\mathbb{R}$.
		\item $\sum\limits_{n = 1}^{\infty} \frac{(-1)^n}{x^2 + n^3}$ in $\mathbb{R}$.
		\item $\sum\limits_{n = 1}^{\infty} \ln \left( 1 + \frac{1}{n^2 + x^2} \right)$ in $\mathbb{R}$.
		\item $\sum\limits_{n = 1}^{\infty} \frac{1}{3^n \sqrt[3]{1 + n^2 x^2}}$ in $\mathbb{R}$.
		\item $\sum\limits_{n = 1}^{\infty} \frac{(-1)^n x^2}{(1 + x^2)^n}$ in $\mathbb{R}$.
	\end{enumerate}
\end{question}

\begin{solution}
	\begin{enumerate}[leftmargin = *]
		\item
			\begin{align*}
				S_k & = \sum\limits_{n = 0}^{k} x^{n + 1} - x^n \\
                                    & = x^{k + 1} - x^0                         \\
                                    & = x^{k + 1} - 1
			\end{align*}
			Therefore
			\begin{align*}
				\lim\limits_{k \to \infty} S_k & = \lim\limits_{k \to \infty} x^{k + 1} - 1
			\end{align*}
			If $0 \le x < 1$,
			\begin{align*}
				\lim\limits_{k \to \infty} S_k & = \lim\limits_{k \to \infty} x^{k + 1} - 1 \\
                                                               & = 0 - 1                                    \\
                                                               & = -1
			\end{align*}
			If $x = 1$,
			\begin{align*}
				\lim\limits_{k \to \infty} S_k & = \lim\limits_{k \to \infty} 1^{k + 1} - 1 \\
                                                               & = 0
			\end{align*}
			Therefore,
			\begin{align*}
				S(x) &=
					\begin{cases}
						-1 & ;\quad 0 \le x < 1 \\
						0  & ;\quad x = 1       \\
					\end{cases}
			\end{align*}
			Therefore, $S_n(x)$ converges pointwise to $S(x)$.\\
			As $S(x)$ is not continuous in $[0,1]$ but all $x^{n + 1} - x^n$ are, the convergence cannot be uniform.
		\item
			\begin{align*}
				S_k & = \sum\limits_{n = 0}^{k} x^n
			\end{align*}
			Therefore,
			\begin{align*}
				\lim\limits_{k \to \infty} S_k & = \lim\limits_{k \to \infty} \sum\limits_{n = 0}^{k} x^n \\
                                                               & = \frac{x^{k + 1} - 1}{x - 1}
			\end{align*}
			If $0 \le x < 1$,
			\begin{align*}
				\lim\limits_{k \to \infty} S_k & = \lim\limits_{k \to \infty} \frac{x^{k + 1} - 1}{x - 1} \\
                                                               & = \lim\limits_{k \to \infty} \frac{-1}{x - 1}            \\
                                                               & = 1
			\end{align*}
			If $x = 1$,
			\begin{align*}
				\lim\limits_{k \to \infty} S_k & = \lim\limits_{k \to \infty} \sum\limits_{n = 0}^{k} 1^n \\
                                                               & = \lim\limits_{k \to \infty} k + 1                       \\
                                                               & = \infty
			\end{align*}
			Therefore,
			\begin{align*}
				S(x) &=
					\begin{cases}
						-\frac{1}{x - 1} & ;\quad 0 \le x < 1 \\
						\infty           & ;\quad x = 1       \\
					\end{cases}
			\end{align*}
			Therefore, $S_n(x)$ does not converge pointwise to $S(x)$ as $S(x)$ is not defined at $x = 1$.\\
			Hence, there is no uniform convergence.
		\item
			\begin{align*}
				\lim\limits_{n \to \infty} \frac{1}{x^2 + n} & = 0
			\end{align*}
			Therefore, as $\sum\limits_{n = 1}^{\infty} \frac{(-1)^n}{x^2 + n}$ is a Leibniz series, and as $\lim\limits_{n \to \infty} \frac{1}{x^2 + n} = 0$, the series converges pointwise.
			\begin{align*}
				\left| \frac{(-1)^n}{x^2 + n} \right| & \le \frac{1}{n}
			\end{align*}
			Therefore, by the Weierstrass M-test, as $\sum \frac{1}{n}$ converges, the series converges uniformly on $\mathbb{R}$.
		\item
			\begin{align*}
				\lim\limits_{n \to \infty} \frac{1}{x^2 + n^3} & = 0
			\end{align*}
			Therefore, as $\sum\limits_{n = 1}^{\infty} \frac{(-1)^n}{x^2 + n^3}$ is a Leibniz series, and as $\lim\limits_{n \to \infty} \frac{1}{x^2 + n^3} = 0$, the series converges pointwise.
			\begin{align*}
				\left| \frac{(-1)^n}{x^2 + n^3} \right| & \le \frac{1}{n^3}
			\end{align*}
			Therefore, by the Weierstrass M-test, as $\sum \frac{1}{n^3}$ converges, the series converges uniformly on $\mathbb{R}$.
		\item
			\begin{align*}
				\left| \ln \left( 1 + \frac{1}{n^2 + x^2} \right) \right| & \le \frac{1}{n^2 + x^2} \\
				\therefore \ln \left( 1 + \frac{1}{n^2 + x^2} \right)     & \le \frac{1}{n^2}
			\end{align*}
			Therefore, by the Weierstrass M-test, as $\sum \frac{1}{n^2}$ converges, the series converges uniformly on $\mathbb{R}$.\\
			Hence, the series also converges pointwise on $\mathbb{R}$.
		\item
			\begin{align*}
				\left| \frac{1}{3^n \sqrt[3]{1 + n^2 x^2}} \right| & \le \frac{1}{3^n}
			\end{align*}
			Therefore, by the Weierstrass M-test, as $\sum \frac{1}{3^n}$ converges, the series converges uniformly on $\mathbb{R}$\\
			Hence, the series also converges pointwise on $\mathbb{R}$.
		\item
			\begin{align*}
				\lim\limits_{n \to \infty} \frac{x^2}{(1 + x^2)^n} & = 0
			\end{align*}
			Therefore, as $\sum\limits_{n = 1}^{\infty} \frac{(-1)^n x^2}{(1 + x^2)^n}$ is a Leibniz series, and as $\lim\limits_{n \to \infty} \frac{1}{x^2 + n} = 0$, the series converges pointwise.
			\begin{align*}
				\sup\limits_{\mathbb{R}} \left| f_n(x) - f(x) \right| & = \sup\limits_{\mathbb{R}} \left| \frac{x^2}{(1 + x^2)^n} - 0 \right| \\
                                                                                      & = \sup\limits_{\mathbb{R}} \frac{x^2}{(1 + x^2)^n}
			\end{align*}
			Therefore, differentiating, the critical points are
			\begin{align*}
				x & = 0 \\
				x & = \pm \frac{1}{\sqrt{n^2 + 1}}
			\end{align*}
			Therefore, the maximum value of the function is at $x = \pm \frac{1}{\sqrt{n^2 + 1}}$.\\
			Therefore,
			\begin{align*}
				\lim\limits_{n \to \infty} \sup\limits_{\mathbb{R}} |f_n(x) - f(x)| & = \lim\limits_{n \to \infty} \frac{\frac{1}{n^2 + 1}}{\left( 1 + \frac{1}{n^2 + 1} \right)^2} \\
                                                                                                    & = 0
			\end{align*}
			Therefore, the convergence is uniform.
	\end{enumerate}
\end{solution}

\begin{question}
	Let $\{f_n(x)\}$ be a sequence of functions defined in the domain $I$.
	\begin{enumerate}
		\item Prove that if the series $\sum\limits_{n = 1}^{\infty} |f_n(x)|$ converges uniformly on $I$ then $\sum\limits_{n = 0}^{\infty} f_n(x)$ converges uniformly on $I$.
		\item Show that the converse is not true, i.e. uniform convergence of $\sum\limits_{n = 0}^{\infty} f_n(x)$ does not imply uniform convergence of $\sum\limits_{n = 0}^{\infty} |f_n(x)|$.
	\end{enumerate}
\end{question}

\begin{solution}
	\begin{enumerate}
		\item
			As $\sum |f_n(x)|$ converges uniformly,
			\begin{align*}
				\lim\limits_{k \to \infty} \sum\limits_{n = 1}^{k} |f_n(x)| & = 0
			\end{align*}
			Therefore, as $|f_n(x)| = \pm f_n(x)$,
			\begin{align*}
				\lim\limits_{k \to \infty} \sum\limits_{n = 1}^{k} f_n(x)            & = \pm \lim\limits_{k \to \infty} \sum\limits_{n = 1}^{k} |f_n(x)| \\
				\therefore \lim\limits_{k \to \infty} \sum\limits_{n = 1}^{k} f_n(x) & = \pm 0                                                           \\
				\therefore \lim\limits_{k \to \infty} \sum\limits_{n = 1}^{k} f_n(x) & = 0
			\end{align*}
			Therefore, $\sum f_n(x)$ converges uniformly on $I$.
			\qed
		\item
			Let
			\begin{align*}
				f_n(x)            & = \frac{(-1)^n}{n} \\
				\therefore f_n(x) & = \frac{1}{n}
			\end{align*}
			Therefore, $\sum \frac{(-1)^n}{n}$ converges, but $\sum \frac{1}{n}$ diverges.\\
			Hence, uniform convergence of $\sum\limits_{n = 0}^{\infty} f_n(x)$ does not imply convergence of $\sum\limits_{n = 0}^{\infty} |f_n(x)|$.
			\qed
	\end{enumerate}
\end{solution}

\begin{question}
	Let $f(x) = \sum\limits_{n = 1}^{\infty} (-1)^n \frac{\cos\left( \frac{x}{n} \right)}{n^2 + 1}$.
	Show that $f(x)$ is continuous on $\mathbb{R}$.
	Is it possible to differentiate $f(x)$ term by term?
\end{question}

\begin{solution}
	\begin{align*}
		\lim\limits_{n \to \infty} \frac{\cos\left( \frac{x}{n} \right)}{n^2 + 1} & = 0
	\end{align*}
	Therefore, as $\sum\limits_{n = 1}^{\infty} (-1)^n \frac{\cos\left( \frac{x}{n} \right)}{n^2 + 1}$ is a Leibniz series, and as $\lim\limits_{n \to \infty} \frac{\cos\left( \frac{x}{n} \right)}{n^2 + 1} = 0$, the series converges pointwise.
	\begin{align*}
		\left| \frac{\cos\left( \frac{x}{n} \right)}{n^2 + 1} \right|            & \le \frac{1}{n^2 + 1} \\
		\therefore \left| \frac{\cos\left( \frac{x}{n} \right)}{n^2 + 1} \right| & \le \frac{1}{n^2}
	\end{align*}
	Therefore, by the Weierstrass M-test, as $\sum \frac{1}{n^2}$ converges, the series converges uniformly.
	Therefore, the limit function $f(x)$ is continuous.
	\begin{align*}
		\dod{}{x} \left( \frac{\cos\left( \frac{x}{n} \right)}{n^2 + 1} \right) & = \frac{-\frac{1}{n} \sin\left( \frac{x}{n} \right)}{n^2 + 1} \\
                                                                                        & = -\frac{\sin\left( \frac{x}{n} \right)}{n^3 + n}
	\end{align*}
	As the derivative exists and is continuous on $\mathbb{R}$, it is possible to differentiate $f(x)$ term by term.
\end{solution}

\begin{question}
	Define $f(x) = \sum\limits_{n = 0}^{\infty} \frac{x^n}{2 + n}$.
	Find the domain of convergence of this series.
	In what domain can we use term by term differentiation to show that $\left( x^2 f(x) \right)' = \frac{x}{1 - x}$?
\end{question}

\begin{solution}
	\begin{align*}
		R & = \lim\limits_{n \to \infty} \left| \frac{a_n}{a_{n + 1}} \right|     \\
                  & = \lim\limits_{n \to \infty} \left| \frac{2 + (n + 1)}{2 + n} \right| \\
                  & = \lim\limits_{n \to \infty} \left| \frac{n + 3}{n + 2} \right|       \\
                  & = 1
	\end{align*}
	If $x = -1$,
	\begin{align*}
		f(x) & = \sum\limits_{n = 0}^{\infty} \frac{(-1)^n}{2 + n}
	\end{align*}
	Therefore, as the series is a Leibniz series, and as $\lim\limits_{n \to \infty} \frac{1}{2 + n} = 0$, the series converges pointwise.
	If $x = 1$,
	\begin{align*}
		f(x) & = \sum\limits_{n = 0}^{\infty} \frac{1^n}{2 + n} \\
                     & = \sum\limits_{n = 0}^{\infty} \frac{1}{2 + n}
	\end{align*}
	Therefore, the series diverges.\\
	Therefore, the domain of convergence is $[-1,1)$.\\
	\begin{align*}
		\dod{}{x} \left( \frac{x^2}{2 + n} \right) & = \frac{2 x}{2 + n}
	\end{align*}
	As the derivative is continuous on $[-1,1)$ and the series converges in $[-1,1)$, we can use term by term differentiation in $[-1,1)$.
\end{solution}

\end{document}
