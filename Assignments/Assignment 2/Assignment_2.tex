\documentclass[fleqn, a4paper, 12pt, oneside]{amsart}
\usepackage{exsheets, tasks}
\usepackage{amsmath, amssymb, amsthm} %standard AMS packages
\usepackage{marginnote} %marginnotes
\usepackage{gensymb} %miscellaneous symbols
\usepackage{commath} %differential symbols
\usepackage{xcolor} %colours
\usepackage{cancel} %cancelling terms
\usepackage{siunitx} %formatting units
\usepackage{tikz, pgfplots} %diagrams
\usetikzlibrary{calc, hobby, patterns, intersections}
\usepackage{graphicx} %inserting graphics
\usepackage{hyperref} %hyperlinks
\usepackage{datetime} %date and time
\usepackage{ulem} %underline for \emph{}
\usepackage{xfrac} %inline fractions
\usepackage{enumerate} %numbered lists
\usepackage{float} %inserting floats

\newcommand\numberthis{\addtocounter{equation}{1}\tag{\theequation}} %adds numbers to specific equations in non-numbered list of equations

\newcommand{\AxisRotator}[1][rotate=0]{
	\tikz [x=0.25cm,y=0.60cm,line width=.2ex,-stealth,#1] \draw (0,0) arc (-150:150:1 and 1);%
} %rotation symbols on axes

\theoremstyle{definition}
\newtheorem{example}{Example}
\newtheorem{definition}{Definition}

\theoremstyle{theorem}
\newtheorem{theorem}{Theorem}

\newcommand{\curl}{\mathrm{curl\,}}

\makeatletter
\@addtoreset{section}{part} %resets section numbers in new part
\makeatother

\renewcommand{\thesubsection}{(\arabic{subsection})}
\renewcommand{\thesection}{(\arabic{section})}

%section headings on left
\makeatletter
\def\specialsection{\@startsection{section}{1}%
	\z@{\linespacing\@plus\linespacing}{.5\linespacing}%
	%  {\normalfont\centering}}% DELETED
	{\normalfont}}% NEW
\def\section{\@startsection{section}{1}%
	\z@{.7\linespacing\@plus\linespacing}{.5\linespacing}%
	%  {\normalfont\scshape\centering}}% DELETED
	{\normalfont\scshape}}% NEW
\makeatother

%forces newline after subsection
\makeatletter
\def\subsection{\@startsection{subsection}{3}%
	\z@{.5\linespacing\@plus.7\linespacing}{.1\linespacing}%
	{\normalfont\itshape}}
\makeatother

\settasks{counter-format = tsk[1].}

\SetupExSheets{solution/print = true}

%opening
\title
{
	Differential and Integral Calculus\\
	Assignment 2
}
\author
{
	Aakash Jog\\
	ID : 989323563
}
\date{\formatdate{26}{3}{2015}}

\begin{document}
	
\maketitle
%\setlength{\mathindent}{0pt}

\begin{question} % % % % % % % % % %Change tasks to (a)
	Find the following limits (using the sandwich theorem)
	\begin{tasks}
		\task $\dfrac{1}{n^2} + \dfrac{1}{(n + 1)^2} + \dots + \dfrac{1}{(2n)^2}$
		\task $\dfrac{\sin n}{n + \cos n}$
		\task $\sqrt[n]{3n - \sqrt{n}}$
	\end{tasks}
\end{question}

\begin{solution}
	\begin{tasks}
		\task
			\begin{gather*}
				0 \le \dfrac{1}{n^2} + \dots + \dfrac{1}{(2n)^2} \le n \cdot \dfrac{1}{n^2}\\
				\therefore \lim\limits_{n \to \infty} 0 \le \lim\limits_{n \to \infty} \dfrac{1}{n^2} + \dots + \dfrac{1}{(2n)^2} \le \lim\limits_{n \to \infty} \dfrac{1}{n}\\
				\therefore 0 \le \lim\limits_{n \to \infty} \dfrac{1}{n^2} + \dots + \dfrac{1}{(2n)^2} \le 0
			\end{gather*}
			Therefore, by the Sandwich Theorem,
			\begin{equation*}
				\lim\limits_{n \to \infty} \dfrac{1}{n^2} + \dots + \dfrac{1}{(2n)^2} = 0
			\end{equation*}
		\task
			\begin{gather*}
				\dfrac{-1}{n + 1} \le \dfrac{\sin n}{n + \cos n} \le \dfrac{1}{n - 1}\\
				\therefore \lim\limits_{n \to \infty} \dfrac{-1}{n + 1} \le \lim\limits_{n \to \infty} \dfrac{\sin n}{n + \cos n} \le \lim\limits_{n \to \infty} \dfrac{1}{n - 1}\\
				\therefore 0 \le \lim\limits_{n \to \infty} \dfrac{\sin n}{n + \cos n} \le 0
			\end{gather*}
			Therefore, by the Sandwich Theorem,
			\begin{equation*}
				\lim\limits_{n \to \infty}  \dfrac{\sin n}{n + \cos n} = 0
			\end{equation*}
		\task
			\begin{gather*}
				\sqrt[n]{3n - n} \le \sqrt[n]{3n - \sqrt{n}} \le \sqrt[n]{3n}\\
				\therefore \lim\limits_{n \to \infty} \sqrt[n]{3n - n} \le \lim\limits_{n \to \infty} \sqrt[n]{3n - \sqrt{n}} \le \lim\limits_{n \to \infty} \sqrt[n]{3n}\\
				\therefore \lim\limits_{n \to \infty} \sqrt[n]{2n} \le \lim\limits_{n \to \infty} \sqrt[n]{3n - \sqrt{n}} \le \lim\limits_{n \to \infty} \sqrt[n]{3n}\\
				\therefore \lim\limits_{n \to \infty} 2^{\sfrac{1}{n}} \sqrt[n]{n} \le \lim\limits_{n \to \infty} \sqrt[n]{3n - \sqrt{n}} \le \lim\limits_{n \to \infty} 3^{\sfrac{1}{n}} \sqrt[n]{n}\\
				\therefore 1 \le \lim\limits_{n \to \infty} \sqrt[n]{3n - \sqrt{n}} \le 1
			\end{gather*}
			Therefore, by the Sandwich Theorem,
			\begin{equation*}
				\lim\limits_{n \to \infty} \sqrt[n]{3n - \sqrt{n}} = 1
			\end{equation*}
	\end{tasks}
\end{solution}

\begin{question}
	Let $a,b > 0$. Find the limit $\lim\limits_{n \to \infty} \sqrt[n]{a^n + b^n}$.
\end{question}

\begin{solution}
	If $a > b$,
	\begin{gather*}
		\sqrt[n]{a^n} \le \sqrt[n]{a^n + b^n} \le \sqrt[n]{2a^n}\\
		\therefore \lim\limits_{n \to \infty} \sqrt[n]{a^n} \le \lim\limits_{n \to \infty} \sqrt[n]{a^n + b^n} \le \lim\limits_{n \to \infty} \sqrt[n]{2a^n}\\
		\therefore a \le \lim\limits_{n \to \infty} \sqrt[n]{a^n + b^n} \le a
	\end{gather*}
	Therefore, by the Sandwich Theorem,
	\begin{equation*}
		\lim\limits_{n \to \infty} \sqrt[n]{a^n + b^n} = a
	\end{equation*}
	Similarly, if $b > a$,
	\begin{equation*}
		\lim\limits_{n \to \infty} \sqrt[n]{a^n + b^n} = b
	\end{equation*}
\end{solution}

\begin{question}
	 Check whether the following sequence are bounded from above or from below (or both): (keep in mind that a convergent sequence is always bounded).
	 \begin{tasks}
	 	\task $a_n = \dfrac{n^2 + 1}{n + 2}$
	 	\task $a_n = \dfrac{n^5}{2^n}$
	 	\task $a_n = \sqrt{n^2 - n} - \sqrt{n}$
	 	\task $a_n = \tan \left( \dfrac{\pi}{2} - \dfrac{1}{n} \right)$
	 \end{tasks}
\end{question}

\begin{solution}
	\begin{tasks}
		\task
			\begin{align*}
				a_n &= \dfrac{n^2 + 1}{n + 2}\\
				&= (n - 2) + \dfrac{5}{n + 2}
			\end{align*}
			Therefore, for $n \ge 1$, $\{a_n\}$ is monotonically increasing.\\
			Therefore, the smallest term is
			\begin{align*}
				a_1 &= \dfrac{1 + 1}{1 + 2}\\
				&= \dfrac{2}{3}
			\end{align*}
			Therefore, the series is bounded from below by $\dfrac{2}{3}$.\\
			As the sequence is monotonically increasing, it is not bounded from above.
		\task 
			\begin{align*}
				a_n &= \dfrac{n^5}{2^n}
			\end{align*}
			Let
			\begin{align*}
				f(x) &= \dfrac{x^5}{2^x}
			\end{align*}
			Differentiating and maximizing,
			\begin{align*}
				f(x)_{\textnormal{max}} &= \dfrac{5}{\ln 2}
			\end{align*}
			Therefore, as $f(x)$ is bounded from above by $\dfrac{5}{\ln 2}$, $\{a_n\}$ is also bounded from above be $\dfrac{5}{\ln 2}$.\\
			~\\
			\begin{align*}
				\lim\limits_{x \to \infty} f(x) &= 0\\
				\therefore \lim\limits_{n \to \infty} a_n &= 0
			\end{align*}
			Therefore, $\{a_n\}$ is bounded from below by $0$.
		\task
			\begin{align*}
				a_n &= \sqrt{n^2 - n} - \sqrt{n}
			\end{align*}
			Let
			\begin{align*}
				f(x) &= \sqrt{x^2 - x} - \sqrt{x}\\
				\therefore f'(x) &= \dfrac{1}{2} \left( \dfrac{2x - 1}{\sqrt{x (x  -1)}} - \dfrac{1}{\sqrt{x}} \right)
			\end{align*}
			Therefore, minimizing,
			\begin{align*}
				f(x)_{\textnormal{min}} &= -1
			\end{align*}
			Therefore, $f(x)$ has minimum value $-1$, but no maximum value.
			Therefore, $\{a_n\}$ is bounded from below by $-1$.
		\task
			\begin{align*}
				a_n = \tan \left( \dfrac{\pi}{2} - \dfrac{1}{n} \right)
			\end{align*}
			Let
			\begin{align*}
				f(x) &= \tan \left( \dfrac{\pi}{2} - \dfrac{1}{x} \right)\\
				&= \cot \dfrac{1}{n}
			\end{align*}
			Therefore, as $\dfrac{1}{n}$ is monotonically decreasing $\cot \dfrac{1}{n}$ is monotonically increasing.\\
			Therefore, the sequence is not bounded from above.
			The minimum value of the sequence is
			\begin{align*}
				a_1 &= \cot 1
			\end{align*}
			Therefore, the sequence is bounded from below by $\cot 1$.
	\end{tasks}
\end{solution}

\begin{question}
	Check whether the following sequences are eventually monotone (i.e whether there exists $N \in \mathbb{N}$ such that $a_n$ is monotone for all $n > N$.
	\begin{tasks}
		\task $a_n = \sqrt{n} - \dfrac{1}{n}$
		\task $a_n = \sin (\pi n)$
	\end{tasks}
\end{question}

\begin{solution}
	\begin{tasks}
		\task
			\begin{align*}
				a_n &= \sqrt{n} - \dfrac{1}{n}
			\end{align*}
			Let
			\begin{align*}
				f(x) &= \sqrt{x} - \dfrac{1}{x}\\
				\therefore f'(x) &= \dfrac{1}{x^2} + \dfrac{1}{2 \sqrt{x}}
			\end{align*}
			Therefore, $f(x)$ is monotonically increasing on $(0,\infty)$.\\
			Therefore, $\{a_n\}$ is monotonically increasing for all $n > 1$.
		\task
			\begin{align*}
				a_n &= \sin (\pi n)\\
				\therefore \{a_n\} &= \sin (\pi), \sin (2 \pi), \sin (3 \pi), \dots\\
				&= 0, 0, 0, \dots
			\end{align*}
			Therefore, for all $n \ge 1$, the sequence is monotonically increasing.
	\end{tasks}
\end{solution}

\begin{question}
	Prove that the following sequences converge and find their limits
	\begin{tasks}
		\task $a_1 = \sqrt{2}$, $a_{n + 1} = \sqrt{2 + a_n}$
		\task $a_1 = 2$, $a_{n + 1} = \sqrt{2 a_n - 1}$
		\task $a_1 = 2$, $a_{n + 1} = \dfrac{1}{2} \left( a_n + \dfrac{1}{a_n} \right)$
	\end{tasks}
\end{question}

\begin{solution}
	\begin{tasks}
		\task
			\begin{align*}
				a_2 &= \sqrt{2 + \sqrt{2}}
				&\le \sqrt{2}\\
				\therefore a_2 &\le a_1
			\end{align*}
			If possible, let $a_{n - 1} \le a_n$.\\
			Therefore,
			\begin{align*}
				a_n &= \sqrt{2 + a_{n - 1}}\\
				&\le \sqrt{2 + a_n}\\
				\therefore a_n &\le a_{n + 1}
			\end{align*}
			Therefore, by induction, the sequence is monotonically increasing.\\
			~\\
			If possible, let 
			\begin{equation*}
				\lim\limits_{n \to \infty} a_n = l \ge a_1
			\end{equation*}
			Therefore,
			\begin{align*}
				\lim\limits_{n \to \infty} a_n &= \lim\limits_{n \to \infty} \sqrt{2 + a_{n - 1}}\\
				\therefore l &= \sqrt{2 + l}\\
				\therefore l &= 2
			\end{align*}
			~\\
			\begin{align*}
				a_1 &= \sqrt{2}\\
				\therefore a_1 &\le l
			\end{align*}
			If possible, let $a_n \le l$.\\
			Therefore,
			\begin{align*}
				a_{n + 1} &= \sqrt{2 + a_n}\\
				&\le \sqrt{2 + l}\\
				\therefore a_{n + 1} &\le l
			\end{align*}
			Therefore, by induction, the sequence is bounded from above.\\
			~\\
			Therefore, the sequence is monotonically increasing and bounded from above by $l = 2$.\\
			Therefore, it converges to $l = 2$.
		\task
			\begin{align*}
				a_2 &= \sqrt{2a_1 - 1}\\
				&= \sqrt{3}\\
				\therefore a_2 &\le a_1
			\end{align*}
			If possible, let $a_{n - 1} \le a_n$.\\
			Therefore,
			\begin{align*}
				a_n &= \sqrt{2 a_{n - 1} - 1}\\
				&\ge \sqrt{2 a_n - 1}\\
				\therefore a_n &\ge a_{n + 1}
			\end{align*}
			Therefore, by induction, the sequence is monotonically decreasing.\\
			~\\
			If possible, let 
			\begin{equation*}
				\lim\limits_{n \to \infty} a_n = l
			\end{equation*}
			Therefore,
			\begin{align*}
				\lim\limits_{n \to \infty} a_n &= \lim\limits_{n \to \infty} \sqrt{2 a_{n - 1} - 1}\\
				\therefore l &= \sqrt{2l - 1}\\
				\therefore l &= 1
			\end{align*}\\
			~\\
			\begin{align*}
				a_1 &= 2\\
				&\ge 1\\
				\therefore a_1 &\ge l
			\end{align*}
			If possible, let $a_{n} \ge l$.\\
			Therefore,
			\begin{align*}
				a_{n + 1} &= \sqrt{2 a_n - 1}\\
				&\ge \sqrt{2l - 1}\\
				&\ge \sqrt{2 - 1}\\
				&\ge 1\\
				\therefore a_{n + 1} \ge l
			\end{align*}
			Therefore, as the sequence is monotonically decreasing and is bounded from below by $l = 1$, it converges to $l = 1$.
		\task
			\begin{align*}
				a_2 &= \dfrac{1}{2} \left( a_1 + \dfrac{1}{a_1} \right)\\
				&= \dfrac{1}{2} \left( 2 + \dfrac{1}{2} \right)\\
				&= \dfrac{5}{4}\\
				\therefore a_2 &\le a_1
			\end{align*}
			If possible let $a_{n - 1} \ge a_n$.\\
			Therefore,
			\begin{align*}
				a_n &= \dfrac{1}{2} \left( a_{n - 1} + \dfrac{1}{a_{n - 1}} \right)\\
				&\ge \dfrac{1}{2} \left( a_n + \dfrac{1}{a_n} \right)\\
				\therefore a_n &\ge a_{n + 1}
			\end{align*}
			Therefore, by induction, the sequence is monotonically decreasing.\\
			~\\
			If possible, let 
			\begin{equation*}
				\lim\limits_{n \to \infty} a_n = l
			\end{equation*}
			Therefore,
			\begin{align*}
				\lim\limits_{n \to \infty} a_n &= \lim\limits_{n \to \infty} \dfrac{1}{2} \left( a_n + \dfrac{1}{a_n} \right)\\
				\therefore l &= \dfrac{1}{2} \left( l + \dfrac{1}{l} \right)\\
				\therefore l &= 1
			\end{align*}
			~\\
			\begin{align*}
				a_1 &= 2\\
				&\ge 1\\
				\therefore a_1 &\ge l
			\end{align*}
			If possible, let $a_n \ge l$.\\
			Therefore,
			\begin{align*}
				a_{n + 1} &= \dfrac{1}{2} \left( a_n + \dfrac{1}{a_n} \right)\\
				&\ge \dfrac{1}{l + \dfrac{1}{l}}\\
				\therefore a_{n + 1} &\ge l
			\end{align*}
			Therefore, by induction, the sequence is bounded from below.\\
			Therefore, as the sequence is monotonically decreasing and is bounded from below by $l = 1$, it converges to $l = 1$.
	\end{tasks}
\end{solution}

\begin{question}
	Prove or disprove: If $a_n$ and $b_n$ are bounded sequences then $a_n b_n$ is bounded.
\end{question}

\begin{solution}
	Let
	\begin{equation*}
		a \le a_n \le A
	\end{equation*}
	and
	\begin{equation*}
		b \le b_n \le B
	\end{equation*}
	Therefore,
	\begin{equation*}
		a_n b \le a_n b_n \le a_n B
	\end{equation*}
	and 
	\begin{equation*}
		b_n a \le a_n b_n \le b_n A
	\end{equation*}
	Therefore, 
	\begin{equation*}
		\min \{a_n b, b_n a\} \le a_n b_n \le \max \{a_n B, b_n A\}
	\end{equation*}
	Therefore, $a_n b_n$ is bounded.
\end{solution}

%\begin{question}
%	Is there a sequence $a_n$ such that $a_n \to 0$ and $\lim\limits_{n \to \infty} (n |a_n - a_{n+1}|) = \infty$?
%\end{question}
%
%\begin{solution}
%	
%\end{solution}

\end{document}