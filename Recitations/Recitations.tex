\documentclass[fleqn, a4paper, 12pt, twoside]{article}
\usepackage{exsheets}
\usepackage{amsmath, amssymb, amsthm} %standard AMS packages
\usepackage{marginnote} %marginnotes
\usepackage{gensymb} %miscellaneous symbols
\usepackage{commath} %differential symbols
\usepackage{xcolor} %colours
\usepackage{cancel} %cancelling terms
\usepackage{siunitx} %formatting units
\usepackage{tikz, pgfplots} %diagrams
	\usetikzlibrary{calc, hobby, patterns, intersections}
\usepackage{graphicx} %inserting graphics
\usepackage{hyperref} %hyperlinks
\usepackage{datetime} %date and time
\usepackage{ulem} %underline for \emph{}
\usepackage{xfrac, lmodern} %inline fractions
\usepackage{enumerate} %numbered lists
\usepackage{float} %inserting floats
\usepackage{circuitikz} %circuit diagrams

\newcommand\numberthis{\addtocounter{equation}{1}\tag{\theequation}} %adds numbers to specific equations in non-numbered list of equations

\newcommand{\AxisRotator}[1][rotate=0]{
	\tikz [x=0.25cm,y=0.60cm,line width=.2ex,-stealth,#1] \draw (0,0) arc (-150:150:1 and 1);%
} %rotation symbols on axes

\theoremstyle{definition}
\newtheorem{example}{Example}
\newtheorem{definition}{Definition}

\theoremstyle{theorem}
\newtheorem{theorem}{Theorem}

\newcommand{\curl}{\mathrm{curl\,}}

\makeatletter
\@addtoreset{section}{part} %resets section numbers in new part
\makeatother

%opening
\title{Differential and Integral Calculus : Recitations}
\author{Aakash Jog}
\date{2014-15}

\begin{document}

\maketitle
%\setlength{\mathindent}{0pt}

\tableofcontents

\newpage
\section{Instructor Information}

\textbf{Michael Bromberg}\\
~\\
E-mail: \href{mailto:micbromberg@gmail.com}{micbromberg@gmail.com}\\

\newpage

\part{Sequences and Series}

\section{Sequences}

\begin{question}
	Prove:
	\begin{equation*}
		\lim\limits_{n \to \infty} \dfrac{2n^2 + n + 1}{n^2 + 3} = 2
	\end{equation*}
\end{question}

\begin{solution}[print]
	Let
	\begin{equation*}
		\varepsilon > 0
	\end{equation*}

	\begin{align*}
		\left| \dfrac{2n^2 + n + 1}{n^2 + 3} - 2 \right| &= \left| \dfrac{2n^2 + n + 1 - 2n^2 - 6}{n^2 + 3} \right|\\
		&= \left| \dfrac{n - 5}{n^2 + 3} \right| \\
		&\leq \left| \dfrac{n - 5}{n^2} \right|\\
		&\leq \dfrac{1}{n}\\
		&< \varepsilon
	\end{align*}
	Therefore, let $N = \left[ \dfrac{1}{\varepsilon} \right] + 1$.
	Hence, for this $N$, $|a_n - L| < \varepsilon$.\\
	Therefore, $\lim\limits_{n \to \infty} \dfrac{2n^2 + n + 1}{n^2 + 3} = 2$.
	\qed
\end{solution}

\begin{question}
	Prove
	\begin{equation*}
		\lim\limits_{n \to \infty} \dfrac{n^3 + \sin n + n}{2n^4} = 0
	\end{equation*}
\end{question}

\begin{solution}[print]
	Let $\varepsilon > 0$
	\begin{align*}
		\left| \dfrac{n^3 + \sin n + n}{2n^4} \right| &\leq \left| \dfrac{n^3 + 1 + n}{2n ^4} \right|\\
		&\leq \left| \dfrac{3n^3}{2n^4} \right| = \dfrac{3}{2} \cdot \dfrac{1}{n} < \varepsilon
	\end{align*}
	Therefore, let $N = \left[ \dfrac{3}{2 \varepsilon} \right] + 1$.
	Hence, for this $N$, $|a_n - L| < \varepsilon$.\\
	Therefore, $\lim\limits_{n \to \infty} \dfrac{n^3 + \sin n + n}{2n^4} = 0$
	\qed
\end{solution}

\begin{question}
	Calculate $\sqrt[3]{n^3 + 3n} - n$.
\end{question}

\begin{solution}[print]
	\begin{align*}
		a^n - b^n = (a - b) \cdot (a^{n - 1} + a^{n - 2} b + \dots + a b^{n - 2} + b^{n - 1})
	\end{align*}
	Therefore, let
	\begin{align*}
		a &= \sqrt[3]{n^3 + 3n}\\
		b &= \sqrt[3]{n^3}
	\end{align*}

	\begin{align*}
		a - b &= \dfrac{a^3 - b^3}{a^2 + a b + b^2}\\
		\therefore \sqrt[3]{n^3 + 3n} - n &= \dfrac{n^3 + 3n - n^3}{(n^3 + 3n)^{\sfrac{2}{3}} + (n^3 + 3n)^{\sfrac{1}{3}} n + n^2}\\
		&= \dfrac{3}{\left( \dfrac{n^3 + 3n}{n^{\sfrac{3}{2}}} \right)^{\sfrac{2}{3}} + \left( \dfrac{n^3 + 3n}{n^3} \right)^{\sfrac{1}{3} n} + n}
	\end{align*}
	Therefore, the limit is 0.
\end{solution}

\begin{question}
	Prove
	\begin{equation*}
		\lim\limits_{n \to \infty} \dfrac{n!}{n^n} = 0
	\end{equation*}
\end{question}

\begin{solution}[print]
	\begin{equation*}
		0 \leq \dfrac{n!}{n^n} = \dfrac{1}{n} \dfrac{2}{n} \dots \dfrac{n}{n} \leq \dfrac{1}{n}\\
	\end{equation*}
	Therefore, by the Sandwich Theorem, $\lim\limits_{n \to \infty} \dfrac{n!}{n^n} = 0$.
\end{solution}

\begin{question}
	Let $a_1 = 3$, $a_{n + 1} = 1 + \sqrt{6 + a_n}$. Prove that $a_n$ converges and find its limit.
\end{question}

\begin{solution}[print]
	If possible, let $\lim\limits_{n \to \infty} a_n = l$.
	\begin{align*}
		a_{n + 1} &= 1 + \sqrt{6 + a_n}\\
		\intertext{Taking the limit on both sides,}
		l &= 1 + \sqrt{6 + l}\\
		\therefore l - 1 &= \sqrt{6 + l}\\
		\therefore l &= \dfrac{3 \pm \sqrt{29}}{2}
	\end{align*}
	As $a_n \geq 0$, $l = \dfrac{3 + \sqrt{29}}{2}$.
	~\\
	\begin{align*}
		a_2 &= 1 + \sqrt{6 + a_1}\\
		&= 1 + \sqrt{6 + 3}\\
		&= 4\\
		\therefore a_2 &> a_1
	\end{align*}
	If possible, let $a_n \geq a_{n - 1}$.\\
	Therefore,
	\begin{align*}
		a_{n + 1} &= 1 + \sqrt{6 + a_n}\\
		&\geq 1 + \sqrt{6 + a_{n + 1}} = a_n
	\end{align*}
	Therefore by induction, $\{a_n\}$ is monotonically increasing.
	~\\
	\begin{align*}
		a_1 &= 3\\
		\therefore a_1 \leq 5
	\end{align*}
	If possible, let $a_n \leq 5$.\\
	Therefore,
	\begin{equation*}
		a_{n + 1} = 1 + \sqrt{6 + a_n} \leq q + \sqrt{11} \leq 5
	\end{equation*}
	Therefore by induction, $\{a_n\}$ is bounded from above by 5.
\end{solution}

\href{tel:+972586283629}{+972 58-628-3629}

\end{document}
